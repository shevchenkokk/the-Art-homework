\documentclass[10pt,pdf,hyperref={unicode}]{beamer}
\beamertemplatenavigationsymbolsempty
\setbeamertemplate{footline}[page number]
% Set it for the internal PhD thesis defence to reduce number of slides
%\setbeamersize{text margin left=0.5em, text margin right=0.5em}

\usepackage[utf8]{inputenc}
%\usepackage[english, russian]{babel}
\usepackage{bm}
\usepackage{multirow}
\usepackage{ragged2e}
\usepackage{indentfirst}
\usepackage{multicol}
\usepackage{subfig}
\usepackage{amsmath,amssymb}
\usepackage{enumerate}
\usepackage{mathtools}
\usepackage{comment}
\usepackage[all]{xy}
\usepackage{tikz}
\usetikzlibrary{positioning,arrows}
\tikzstyle{name} = [parameters]
\definecolor{name}{rgb}{0.5,0.5,0.5}

% colors
\definecolor{darkgreen}{rgb}{0.0, 0.2, 0.13}
\definecolor{darkcyan}{rgb}{0.0, 0.55, 0.55}

%----------------------------------------------------------------------------------------------------------

\title{ Put the title of your thesis \\ here}
%\author{Name Surname}
%\institute[]{}
%\date{2024}

%---------------------------------------------------------------------------------------------------------
\begin{document}


\setcounter{page}{2}%remove here for the title
%----------------------------------------------------------------------------------------------------------
\begin{frame}{Application of Graph Neural Networks to Mitigate Popularity Bias in Content Recommendations}
    Recommender systems often overemphasize popular items, reducing diversity and limiting exposure to less popular but relevant content.

    \begin{block}{The problem}
        How can we leverage GNNs to address the popularity bias in content recommendations ensuring overall recommendation quality
        is not compromised?
    \end{block}

    \begin{block}{The goal of the experiment}
        To analyze how the proposed model compares to existing baselines and determine which models perform best in balancing Precision and Fairness.
    \end{block}

    \begin{block}{The data}
        Four datasets widely used in Popularity Bias research from different subject areas to demonstrate robustness, scalability and reproducibility
        of the models.
    \end{block}
\end{frame}
%----------------------------------------------------------------------------------------------------------
\begin{frame}{Pareto Analysis of Proposed Model and Baselines}

The plot shows the trade-off between Precision and Fairness, highlighting Pareto-optimal models. 

The error: $\scriptstyle{\mathbf{E} = \min_{i \in \text{Pareto}} \sqrt{(\mathbf{P} - \mathbf{P_i})^2 + (\mathbf{F} - \mathbf{F_i})^2}}$.

\begin{columns}
    \begin{column}{0.3\textwidth}
        \begin{enumerate}[1]
            \item Models with similar Precision values can significantly differ in Fairness and vice versa.
            \item Achieving a balance between metrics is an indication of well convergence.
        \end{enumerate}
    \end{column}
    \begin{column}{0.8\textwidth}
	    \includegraphics[width=1\textwidth]{multi_dataset_pareto_comparison}      
    \end{column}
\end{columns}

\bigskip

FairGNN balances accuracy and fairness, providing robust performance in mitigating popularity bias.
\end{frame}
\end{document}