\documentclass[12pt]{article}
\usepackage[utf8]{inputenc}
\usepackage[T1]{fontenc}
\usepackage{amsmath,amsfonts,amssymb}
\usepackage{graphicx}
\usepackage{a4wide}
\title{Comparative analysis of my projects}
%\author{not specified}
\date{}
\begin{document}
\maketitle

\section{Recommender systems using graph neural networks and explicit data}
This project aims to enhance recommender systems by leveraging graph neural networks to model complex user-item interactions using
explicit feedback data such as ratings and reviews. 
\begin{enumerate}
\item \emph{The impact} is the potential improvement in recommendation accuracy and personalization by effectively capturing the intricate
relationships between users and items through graph structures
\item \emph{The consistency} is supported by existing research demonstrating the efficacy of graph neural networks in handling graph-structured data,
with rigorous experiments and analyses validating the approach
\item \emph{The novelty} lies in integrating explicit feedback with graph neural networks in the context of recommender systems, offering a different
perspective from traditional collaborative filtering methods
\item \emph{My contribution} involves designing the graph neural network architecture, implementing the model and conducting experiments to compare
its performance with baseline models
\item \emph{The project focuses} on utilizing graph neural networks to improve the understanding of user preferences and enhance the quality of
recommendations
\end{enumerate}

\section{Popularity bias issues and solutions in recommender systems}
This project investigates the prevalence of popularity bias in recommender systems and explores methods to mitigate its effects to promote diversity
and fairness in recommendations.
\begin{enumerate}
\item \emph{The impact} is the increasing an exposure of less popular items, thereby supporting niche markets and providing users with a broader
range of options
\item \emph{The consistency} is based on documented issues of popularity bias in literature, with analytical and empirical methods used to assess
and address the bias
\item \emph{The novelty} comes from developing or improving algorithms that specifically target popularity bias, possibly introducing new fairness
metrics or adjustment techniques
\item \emph{My contribution} involves conducting a comprehensive analysis of existing approaches to quantify bias, implementing mitigation
strategies and evaluating their effectiveness
\item \emph{The project focuses} on enhancing the fairness and diversity of recommender systems by identifying and correcting popularity-induced
distortions
\end{enumerate}

\section{Reinforcement learning-based recommender systems}
This project explores the application of reinforcement learning to develop recommender systems that adaptively learn from user interactions to
optimize long-term engagement and satisfaction
\begin{enumerate}
\item \emph{The impact} is the advancement of recommendation strategies that proactively adjust to user feedback, potentially leading to more
engaging and personalized user experiences
\item \emph{The consistency} relies on reinforcement learning success in sequential decision-making tasks, supported by theoretical foundations
and practical implementations
\item \emph{The novelty} involves applying reinforcement learning algorithms to the recommendation domain, introducing new reward functions
adapted to user behavior patterns
\item \emph{My contribution} includes providing an up-to-date comprehensive review of reinforcement learning in recommender systems, designing the
model, implementing it and performing experiments to measure improvements over traditional methods
\item \emph{The project focuses} on leveraging reinforcement learning to create dynamic recommendation systems that evolve with user interactions
and receive better performance
\end{enumerate}

\section{Resume}
The project \emph{Recommender systems using graph neural networks and explicit data} has the highest priority since it aligns closely with my
interests in recommender systems, offers significant potential for innovation and allows me to contribute meaningfully to the advancement of
recommender system technologies. In addition, the supervisor who proposed this topic has extensive experience in the field, a PhD in physics and
mathematics, so this is a safer option at the project planning stage.


\end{document}