\documentclass[12pt]{article}
\usepackage[utf8]{inputenc}
\usepackage[T1]{fontenc}
\usepackage{amsmath,amsfonts,amssymb}
\usepackage{graphicx}
\usepackage{a4wide}

\title{Application of Graph Neural Networks to Mitigate Popularity Bias in Content Recommendations}
%\author{not specified, not necessary here}
\date{}
\begin{document}
\maketitle
In this project, we explore the use of Graph Neural Networks (GNNs) to address the issue of popularity bias in content recommendations.
Popularity bias issue leads to the over-representation of popular items, thereby reducing the visibility of less popular but potentially relevant
content. By considering the ability of GNNs to capture complex user-item interactions, we aim to develop a graph-based approach that provides
fair exposure to all items regardless of their popularity. 

\section{Introduction}
Table~\ref{tab:intro_comparative} presents a comparative analysis of GNN-based approaches aimed at mitigating popularity bias in content recommendations.
The table summarizes the strengths and weaknesses of various methods, highlighting the need for a novel solution.

\begin{table}[!htbp]
\label{tab:intro_comparative}
\caption{Comparative analysis of recent approaches to mitigate popularity bias in content recommendations.}
\begin{tabular}{p{5cm}|p{5cm}|p{5cm}}
	Solution & Strengths & Weakness \\
	\hline
	BiGNN: A Bilateral-Branch Graph Neural Network~\cite{Kou2022} & 
	Leverages a bilateral-branch framework to handle both long-tail and popular items, improving fairness & 
	Requires complex optimization and may not generalize well to all recommendation tasks \\
	\hline
	MixGCF: Graph Neural Network-based Recommender Systems~\cite{Huang2021} & 
	Improves GNN-based training by combining graph convolution with user-item interactions, reducing bias & 
	High computational cost and may overfit on sparse or imbalanced data \\
	\hline
	XSimGCL: Simple Graph Contrastive Learning for Recommendation~\cite{Yu2023} & 
	Utilizes a very simple contrastive learning approach that significantly reduces model complexity & 
	Decreases performance on more complex and larger datasets due to its simplicity \\
\end{tabular}
\end{table}
 
 The section references contain the full list, collected for this project. 
\nocite{*} % Remove this to keep the cited referernces only

\bibliographystyle{unsrt}
\bibliography{Kirill-Shevchenko-theArt}
\end{document}